\documentclass[9pt,twoside,a4paper]{article}

\begin{document}
\title{Emco F1P Milling Machine Reference Manual}
\author{Matt Batchelor, Instrument Industries}
\date{February 2018}
\maketitle

\section{Introduction}

Manual milling machines are really useful for removing precise amounts of material from engineering materials such as steels, non-ferrous metals and plastics.

\subsection{Safety}

Using a milling machine safely is a matter of understanding the risks that the machine poses and taking steps to eliminate or minimise those risks.

\textbf{Remember that engineering machinery is designed to quickly remove material from objects that are significantly harder and tougher than the human body. If you or your colleague end up in the path of the spindle, or even the screws that drive the position of the worktable, you or they are likely to end up seriously injured, or worse.}

The main risks you face when using this machine are:

\begin{itemize}
	\item You becoming trapped in the spindle � the rotating part of the machine that holds the cutter.
	\item You becoming trapped and �machined� by the rotating cutter.
	\item You receiving a cut from the sharp edges of the cutter.
	\item You or a fellow member getting injured by flying swarf and chips.
	\item You or a fellow member getting injured by swarf and chips on surfaces around you.
	\item You or a fellow member slipping and/or tripping and impacting the machine. This could be bad if the machine is not running, or really bad if the machine is in operation.
	\item You or a fellow member getting injured by a flying cutter that has come loose from the machine, or bits of a broken cutter that have come away from the spindle.
	\item You or a fellow member getting injured by a flying workpiece or bits of a workpiece that have come loose from the machine.
	\item You or a fellow member getting injured by a flying workpiece or bits of a clamping fixture that have come loose from the machine.
	\item You or a fellow member dropping a heavy workpiece or fixture on your foot.
	\item You and your fellow members breathing in dust, mist or vapour from the cutting process.
	\item The machine toppling over on top of you or a fellow member.
	\item An oblivious fellow member knocks you into the machine.
\end{itemize}

You must always take these safety precautions:

\begin{itemize}
	\item Always wear safety glasses when using this machine.
	\item Always tie back long hair that could get caught in the machine.
	\item Always remove baggy clothing and scarves that could get caught in the machine. If you can�t remove it, wear a workshop jacket or coverall that will contain it.
	\item Wear stout shoes with grippy soles to protect your feet from falling objects, and to help prevent you from slipping on dusty or oily floors.
	\item Do not wear gloves. They can make you more likely to get your hands trapped in the machine.
	\item Keep the workshop clean. Sweep up before, during and after your job if necessary.
	\item Check that the machine is bolted down, and that the vice and any other fixtures are securely bolted to the machine.
	\item Check the condition of the machine, the cutter and any cutter holding or work holding device. Do not use it if it is damaged or incomplete.
	\item Make sure the cutter is sharp and the right geometry for what you are cutting. Blunt and/or incorrect cutters can overload and break, and become lodged in you or your fellow members.
	\item Make sure the area around the machine is clear of stuff. You should be able to jump out of the way of something that has gone wrong without colliding with other objects.
	\item (don�t use the workshop as storage)
	\item Keep your hands clear of the spindle when it is running. If you need clear chips, use a long screwdriver, needle nose pliers, or a paint brush.
	\item Don�t work closely behind someone who is operating a machine tool. You might accidentally nudge them into the machine, with potentially horrible results.
	\item Don�t operate a machine with someone closely behind you. We have laid out the shop so that there is no reason why they should be standing behind you to work. Politely ask them to jog on.
	\item Don�t do dusty work (e.g. wood, tooling board) on these machines. It gets in the slideways and is horrible to clean off.
	\item If your job is so heavy that you are producing a coolant mist, you probably need to outsource it to a bigger machine shop.
\end{itemize}

\section{Basic Principles}

Milling machine � multi-point cutting (tool moves, work held still). Lead screw and slides position and move work with great deal of precision. 

Flexible machine � fundamental part of workshop. Can do a lot of complex work given the right toolholding and workholding technique.
Number of adjustments � always leave it square, never assume its square.

Quill plunge vs Z-travel

\subsection{Dial Gauge}

Flexible tool that is used to locate work and set up machine to run accurately. Dial gauge vs DTI � plunger vs lever. Mounting gauge � magnet stand, flexible arm. Stiffness.

\subsection{Edge finding}

Tap off, read scale, zero, repeat. Remember to account for radius of edge finder.

\section{Toolholding}

\subsection{Chucks and Arbors}

Collet Chuck vs drill chuck. How tapers work. How to change chuck.

\subsection{ER Collet Chuck}

Size -1 mm � can flex (unlike most other collet designs). Don�t use adjustable spanners � use proper wrenches. Less run-out � can be used for drills but can be fiddly for pilot holes.

\subsection{Drill Chuck}

Quicker to change bits. No drawbar � not safe to use for milling.

\subsection{Slitting arbor / shell mill arbor}

Holds saw blades and indexable �shells� for facing and end milling. Slitting � clamp blocks.

\section{Workholding}

\subsection{Milling vice}

Precision and stiffness. Heavier construction than drill press vice � dampen vibrations, keeps jaws square. DON�T DRILL INTO THE VICE.

\subsection{ER40 Collet}

For round / repetitive work. Just like ER20 but bigger. Can lock in vice, centre, and then drop workpiece in.

\subsection{Parallels}

\subsection{V Blocks}

\subsection{Bar Clamps}

Workholding on bed of machine
DON�T DRILL INTO THE BED. Use a spoil board.

Clamp using stepped wedges and clamp bars. Bar should run slightly downhill to work. Screw and nut closer to work than to wedge.

\subsection{1-2-3 Blocks}

\subsection{Machinists' Jacks}

Use machinists jacks or stack of blocks / parallels to shim uneven work.

\end{document}